\newpage
{\Huge \bf Abstract}
\vspace{24pt} 

This topic itself originated from the expansion of a routing problem that was not specifically reasoned in the L11 class. We classify this type of problem as the same problem, reduction. Although we discussed some of the properties of reduction in L11, we found that when we define the reduction we need, the construction of the reduction will conflict with the definition of reduction itself. Therefore, it shows that our definition of the reduction problem does not resolve our actual problem.

When we traced the source, we found reduction from a paper by Wongseelashote in 1979. Although Wongseelashote very skilfully proposed the concept of reduction when discussing path problems, it is regrettable that this paper does not provide detailed structural proof of reduction for its properties.

Therefore, our project is comprised of the following three parts. First of all, the classical reduction is expressed, and we are trying to reasoning the properties of reduction itself. Next, we will try to represent/define reduction in another way, not only to facilitate us to implement, but also to decrease the limitation of the reduction definition to practical problems. After that, we will define a kind of reduction according to requirements, and use this kind of reduction to construct numerous realistic examples. Finally, we will use these practical examples as a combination to solve the path problem mentioned in L11.

\newpage
\vspace*{\fill}
